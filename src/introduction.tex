\section{Задание на практику}
	Разработать модуляцию работы системы (группы) лифтов на языке высокого уравня общего назначения,
		используя тривиальный алгоритм обработки запросов.

	В ходе практики должны быть освоены компетенции:
		\begin{itemize}
			\item способность заниматься научными исследованиями;
			\item способность анализировать и оценивать уровни своих компетенций в сочетании со способностью и готовностью к саморегулированию дальнейшего образования и профессиональной мобильности;
			\item  способность анализировать профессиональную информацию, выделять в ней главное, структурировать, оформлять и представлять в виде аналитических обзоров с обоснованными выводами и рекомендациями.
		\end{itemize}

\newpage
\section{Введение}

Автоматизированные системы управления активно приходят в повседневную жизнь человечества. Сначала, это были системы для управления производственным процессом на крупных предприятиях, теперь данные системы решают и бытовые задачи. Одной из таких задач является доставка человека с одного этажа на другой. Данная задача достаточно подробно описана в книге С.В. Васильева [1], там имеются абстрактная модель, логическая модель и логический вывод. С предложенным логическим выводом справилась бы система автоматического доказательства теорем А.А. Ларионова [2], но реализация всей системы лифтов на позитивно образованных формулах требует больших трудозатрат и будет носить чисто исследовательский характер.

Построить имитационную модель полностью на логическом выводе достаточно сложно, поэтому следует упростить задачу. А значит необходимо разбить модель на две части: систему взаимодействующих объектов и аппарат принятия решений. Таким образом, разработку обоих частей можно вести независимо дуг от друга. Поскольку при разработке имитационной модели можно сразу не иметь готовую логическую часть, а только ту, что будет содержать основные правила и реализовывать первый рассмотренный подход. Получив стабильную модель системы взаимодействия объектов, можно начать разработку и тестирование более сложных алгоритмов логического вывода. В дальнейшем можно будет усложнять имитационную модель, добавлять дополнительные условия и факторы, тем самым приближая модель к реальным условиям.

Основной решаемой задачей является реализация имитационной модели и алгоритма принятия решений. Которые являются системой разделённой на два блока, реализация которых может вестись на разных языках программирования, например: блок имитации на языке питон, сильной стороной которого является большое количество фреймворков, а логический блок на языке пролог, который поддерживает декларативное программирование. Коммуникация этих двух блоков реализуется через интерфейс, что даёт возможность разрабатывать один блок независимо от другого. Предлагаемая модель позволяет не только вести раздельную разработку блоков, а заменять один блок на другой, например блок с реализацией на другом языке. В последствии планируется заменить логический блок на систему автоматического доказательства теорем [2], где поиск логического вывода будет оптимизирован, а имитационный блок на реальную систему лифтов.
