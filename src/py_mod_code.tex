
	Данная симуляция системы управления группой лифтов с ограниченным количеством кабин и несколькими людьми,
		которые приходят для попадания с одного этажа на другой.
		В данной системе используется ресурс для моделирования ограниченного количества кабин.
		Он также определяет процесс выбора кабины и перевозку ей человека.

	Когда человек появляется рядом с шахтой лифта, он вызывает первую из свободных кабин.
		Как только он его кабина забирает, время ожидания человека прекращается.
		Он, наконец, добирается до нужного этажа и уходит.

	После старта системы начинается генерация людей, они появляются после случайного интервала времени,
	пока продолжается симуляция.

\begin{lstlisting}[basicstyle=\footnotesize]
import simpy
import random
import datetime
import os
LOGFILE = 'project.log'
T_INTER = 5
SIM_TIME = 40
NUM_FLOORS = 30
def info(message, env = None):
    now = datetime.datetime.now()
    time = 'null'
    if env is not None :
        time = '%.2f' % (env.now)
    file = open(LOGFILE, 'a')
    #  file.write('%s %s\n' % (NOW.strftime("%Y-%m-%d %H:%M"), message))
    file.write('%s [%s:%s] %s\n'
               % (now.strftime("%d %b %Y %X"), os.getpid(), time, message))
    file.close()
class Man:
    def __init__(self, ind):
        self.name = 'Man %d' % ind
        self.floor = random.randint(0, NUM_FLOORS // 2)
        self.target = self.floor
        while (self.target == self.floor):
            self.target = random.randint(0, NUM_FLOORS)
        info('%s appears on floor \'%s\' with target \'%s\''
             % (self.name, self.floor, self.target), env)
class Cabine(object):
    def __init__(self):
        self.state = 0
        self.floor = 0
class Elevator(object):
    def __init__(self, env, num_machines):
        self.env = env
        self.cab = simpy.Resource(env, num_machines)
        self.cabs = [Cabine() for i in range(num_machines)]
        self.n_cabs = num_machines

    def get_free_cab(self):
        for i in range(self.n_cabs):
            if (self.cabs[i].state == 0):
                return i
    def moving(self, man):
        cid = self.get_free_cab()
        self.cabs[cid].state = 1
        moving_time = abs(man.target - man.floor)
        waiting = abs(man.floor - self.cabs[cid].floor) + moving_time
        yield self.env.timeout(waiting)
        self.cabs[cid].state = 0
        info('%s has reached by Cab %d' % (man.name, cid), env)

def call(env, ind, elev):
    man = Man(ind)
    with elev.cab.request() as request:
        yield request
        info('%s is moving' % (man.name), env)
        yield env.process(elev.moving(man))
def people_generator(env):
    elev = Elevator(env, 2);
    i = 0
    env.process(call(env, i, elev))
    while True:
        yield env.timeout(random.randint(T_INTER - 2, T_INTER + 2))
        i += 1
        env.process(call(env, i, elev))
info('Start session')
env = simpy.Environment()
env.process(people_generator(env))
env.run(until=SIM_TIME)
info('Session end')
\end{lstlisting}

	При том, что данная программа является многопоточной, стоит отметить тот факт, что здесь используются общие ресурсы.
		Эти ресурсы могут использоваться для ограничения количества процессов, использующих их одновременно.
		Процесс должен запрашивать право использования ресурса. Как только право на использование больше не требуется,
		оно должно быть выпущено. Данная система смоделирована как ресурс с ограниченным количеством кабин.
		Люди прибывают к кнопке вызова кабины и просят выполнить перевозку на другой этаж.
		Если все кабины заняты, человек должен ждать, пока один из пассажиров не закончит поездку и не освободит кабину.
