\stepcounter{mysection}\section{\arabic{mysection} Теоретическая часть}

SimPy - это Python фреймворк процессо-ориентированной дискретно-событийной системы моделирования. Его диспетчеры событий основаны на функциях-генераторах Python.

Также они могут использоваться для создания асинхронных сетей или для реализации мультиагентных систем (с как моделируемым, так и реальным взаимодействием).

Процессы в SimPy - это просто Python генераторы, которые используются для моделирования активных компонентов, например, таких как покупатели, транспортные средства или агенты. SymPy также обеспечивает различные виды общих ресурсов для моделирования точек с ограниченной пропускной способностью (например, серверов, касс, тоннелей). Начиная с версии 3.1, SimPy также будет обеспечить возможности мониторинга для помощи в сборе статистических данных о ресурсах и процессах. 

SymPy представляет собой открытую библиотеку символьных вычислений на языке Python. Цель SymPy - стать полнофункциональной системой компьютерной алгебры (CAS), при этом сохраняя код максимально понятным и легко расширяемым. SymPy полностью написан на Python и не требует сторонних библиотек.

SymPy можно использовать не только как модуль, но и как отдельную программу. Программа удобна для экспериментов или для обучения. Она использует стандартный терминал IPython, но с уже включенными в нее важными модулями SymPy и определенными переменными x, y, z.
